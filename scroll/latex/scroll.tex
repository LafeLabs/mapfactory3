
\documentclass[11pt]{article}
\usepackage{hyperref}
\usepackage{graphicx}   
\usepackage[paperheight=8.5in,paperwidth=5.5in,margin=1in]{geometry}

\begin{document}


\subsection{Robots!}\label{robots}

Robots! The word is loaded with both promise and peril. We dream of
robots that do all tedious labor, freeing humanity from it, as well as
of robots that might take over and kill us all(fiction seems to favor
the latter).

I also believe robots can be transformative, although I think we should
look at much of the hype from today's ``tech'' companies with many
grains of salt. Self driving cars and autonomous battle robots have
mostly turned out to be worthless hype machines useful for making
Silicon Valley hucksters rich and not for much else.

Here I will look at some of the robots I think we should build with
Trash Magic which can make a better world for caring for one another and
having adventures, which is what this book is all about.

\subsection{A Rumble of Robots}\label{a-rumble-of-robots}

The collective noun for robots is ``a rumble of robots.'' I'm not sure
where I heard this, I think one of my friends may have made it up, but
it's so perfect it's too good not to use. So I want to talk about
rumbles of robots. In particular the difference between robots used for
consumption and for production.

Amazon is in the process of building robot based infrastructure for
delivery. This is fundamentally a consumption-driven project. The main
initial figure of merit in the growth of their network will be coverage:
the more potential consumers are covered, the better. This will mean
that it is optimal for robots to be as far as possible from other
robots. But how does this picture change for production?

Rumbles of robots are very common on the production side of things.
Those who produce cars and computers and the like often have rumbles of
robots, with humans just as technicians who run the machines. Much like
a cow hand or shepherd, I think there should be a name for those who
herd rumbles of robots: rumbler. So the trash wizard is also a rumbler.
And the trash wizard stick is like the shepherd's crook: a device that
controls a network that consists of your rumble of robots.

That is what seizing the means of production is really all about. It's
not about seizing an existing factory, which will be based on existing
methods, or about building a primitive system that can't compete. It's
about building rumbles of robots which can reproduce themselves by
harvesting free materials to make more, and then rumbling them around to
build what else you need.

Key elements of the trash wizards' robot rumble are mobility and
versatility. They will run off of locally harvested energy, and be
programmed to gather energy as needed as well as materials. They should
scale in the sense that the robots you need for a 10 bot rumble are not
so different from a single roninbot or a 1000 bot rumble. They should be
able to reproduce from found materials and forage for those materials
with some simple guidance from the rumbler.

That is the plan.

\subsection{Robots with different times scales, centuries of work, or
hours of
lifetime}\label{robots-with-different-times-scales-centuries-of-work-or-hours-of-lifetime}

Something that I think needs to be investigated in robot design is time
scale. Capitalists like a certain time scale--the shorter the better.
But without capitalism and its obsession with short term growth and
profits we can have time scales of hundreds of years or even longer in
some cases. Suppose an area of land is contaminated with plutonium or
some other radioactive heavy metal. It might be there for many thousands
of years, making the land uninhabitable. Thousands of years, but not
forever, and plutonium has uses even in a peaceful society without
rules. Why not clean it up?

Perhaps the robots that clean plutonium will grow their own biofuel to
get energy from the sun and slowly pick their way across the land,
working with cyborg worms and fungi to dig up the atoms and move them
together and out of the water table. How many processes of atomic or
molecular transport open up when we allow a process to take thousands of
years? Many. I'm sure capitalists already use the term ``geological
engineering'' but I would say that to truly apply that term, you should
be carrying out a technical/artistic endeavor which takes place on a
geological time scale. That means it has to be \emph{very} easy for
future people to understand, maintain and repair. It also has to
anticipate future geological changes, including catastrophic ones like a
volcano that destroyed all life on earth for a billion years. And it
should have time horizons that stretch well into the 10's of millions of
years. What's your hurry! If we were not all hounded by debt to
capitalists we could take time to really work on hard things like
plutonium cleanup one atom at a time.

\subsection{Earth Robots}\label{earth-robots}

The octahedral ball drone is a octahedron made of three intersecting
sticks, with a flexible joint. Simple mechanical motions of the tips of
the ball-like shape cause it to roll across the landscape with a slight
hopping or walking aspect that makes it able to deal with very rough
terrain.

Rolling ball robots can be used for all sorts of long slow land cleaning
processes. Rather than try to maximize battery life, they will use
capacitors to store energy, and recharge the capacitors from ambient
energy. For a rumble of jacks in the prairie, the obvious source of
power is the wind. Ideally, the wind will be used to create energy which
will immediately go into directed propulsion. This might be slow since
it depends on gusts, but it can go on forever, so slowness becomes ok.
This is technology that you would deploy to spend 1000 years cleaning up
a sacrifice zone, where you want no outside energy or materials to be
needed at all and for the rumble to keep doing its work for hundreds of
years. Also, obviously, clearing of mine fields is an immediate
application. A rumble of tire-sized octahedra could potentially roll
themselves at 10's of miles per hour, keeping up with a car or truck and
making it possible for the rumble to proceed in a mob ahead of a motor
vehicle, taking out IED's in real time. The rumble could end up in a
convoy geometry, stretched out over the length of the road, doing recon
ahead and tracking behind to see what's happening after a convoy passes.
In these applications it probably makes sense for the source of power to
be the trucks or cars in the human/freight convoy, with individuals in
the rumble cycling through the charging station and back out into the
rumble.

These are a great tool for agriculture, or even just gathering. A
gathering rumble could go out and gather roots and berries from the
countryside in a quasi-cultivated area. These roving balls could be
picking up and dropping seeds as they go, mapping where all the useful
plants are, and also harvesting as they go, taking wind, sun and water
as energy sources as needed, then spending energy when it's available to
do the work.

Rolling robots with windmills: they roll, then gather wind electricity
into a capacitor, roll again, and repeat. They can go for hundreds of
miles with no intervention. The instinct to go a certain direction based
on navigating off of the sun is programmed into the physical hardware.
After some long time, maybe many years, the machine calls for help,
eventually someone finds it and follows the instructions for repair and
improvement. With generation after generation editing and helping the
thing exist, it can exist for hundreds of years, slowly cleaning up
wasted sacrifice zones of the old capitalist world.

Free robots like this are a rational response to the fact that the
existing system has created sacrifice zones. These sacrifice zones have
negative economic value in the old system, making them freely available
to be absorbed into the free industrial infrastructure. This is key: in
order to avoid getting crushed by the forces of the old system too early
our movement must exist in the fringes of the current system, where the
old ways have created land of negative value. The very fact that land
can have negative value, that this is a concept that people accept,
should be yet another red flag that assignment of numerical values to
real human values is a morally bankrupt act.

This should always be the goal of free technology if it wants to grow
exponentially without a lot of resistance: the input must be things
deemed of ``negative value'' by the old system. Unlike most projects in
capitalism which constantly drain everyone involved more and more over
time, creating generation after generation of institutional burnout.

\subsection{Air Robots}\label{air-robots}

Everyone is in such a hurry! Most aerial drones for personal use
today(2016) are designed to move very fast for very short periods of
time. Generally with four propellors pointed straight up, they can take
off fast anywhere, go in all directions fast, dodge fast moving
obstacles, and often only last a few minutes. If broken, they have
numerous small parts which can be very hard to fix.

Quad copter personal drones are great capitalist technology: they break
easily, cost a lot, do very little, need constant upgrades, and are
mostly ``useful'' for entertaining the techno-priesthood and annoying
everyone else. Not surprisingly I see much that can be improved here.

The first way I would set about making drones less useless is by making
them float instead of fly with propellors. Given that they're both small
and don't have living cargo, I would say the arguments against hydrogen
for lift are mostly obsolete.



How should motors work for soaring drones? First of all, if the thing is
large enough it can float on the circular current patterns in the upper
atmosphere, holding position with no mechanical work done. But what
about motors for guidance? These motors should be electrostatic, powered
by extremely high voltage giant balloon capacitors which are the main
body of the soaring drone. Using two very light polymers in very thin
sheets with opposite positions on the turboelectric series, it should be
possible when far from the ground to generate \emph{extremely} high
voltages very easily using the mechanical energy source of the rotating
air currents. Electrostatic motors can then run off these, also built
from thin polymer sheets with thin metallization. No magnets and no
copper coils! It's nutty to use the magnetic field for high altitude low
power low speed motors, they should all be based on electric fields,
because it's easy to get megavolts up there.

What about scaling these robots way up in size and weight for use inside
storms? One could imagine giant metal gliders in massive rumbles of 10's
of thousands or maybe even millions of units, all ripping around in a
storm could over the ocean. These then generate giant hydrogen-filled
blimps which then gather in a huge rumble to go turn back into useful
work near a settlement or floating factory.

\subsection{Water Robots}\label{water-robots}

Littoral robot rumbles which can use tides and river currents to
generate electricity to propel themselves upstream should be built. They
can be amphibious, use water to charge and land to move, can move with
hopping, jumping, walking, rolling, and slithering. Littoral trash
cleanup robots are fundable under capitalist government, can make a huge
difference in cleanup of a waterway, and also give us free source
material for more building.

Another robot rumble I want to build is the slithering water robots.
These use the usual magnet and coil arrangement to create a slithering
motion in buoyant objects, which can then smoothly cut through the
water. The fact that this has not been widely deployed is totally
insane: the same drive can be used in reverse to get electrical power
out of wave action. If the length of each robot is a few wavelengths,
the whole thing will be forced into a wave which can create EMF as the
magnets move, which can go into the storage capacitors, then released to
change slightly the nature of the serpentine motion to direct the drone
in a specific direction.

These can be incredibly powerful technology! The ocean can be a
fantastic source of raw materials for the trash wizards. Note that for
neutrally buoyant drones, this can serve to move them through the water
below the surface. One mode of operation might be to cruise a few meters
above the bottom of the ocean, scanning for stuff to salvage, then dive
and grab rocks to be negatively buoyant once a target is found. With
just barely negative buoyancy, the rumble can float just above the
target as they pick it apart. They then drop the weights, rise up,
inflate bags to float(everything is made from rubber, and reversible
air/vacuum/water pumps are in all things), and pull up and bring
material back to assembly centers, which can also be floating robot
rumble factories. With ocean currents and waves as an energy source, and
no hurry, these robots can work as slow as they have to, slowly making
more and more of themselves until they can have a global impact on ocean
cleanup.

The water based propulsion system also is very appealing for boats. I
want a boat that runs on wave action, wind, and tides, to grab energy as
it finds it, and then use it as needed to move toward a destination. I
can imagine this being just about kayak or canoe sized. I could also
imagine a freighter that is meters or even 10's of meters long. That
sounds small for a freighter, but imagine, again, that they're a huge
rumble that can be easily scaled up. This can be a freight swarm to move
materials across water.

\subsection{How Robots Reproduce}\label{how-robots-reproduce}

Not on their own! With help. Robots can always ask for help, and it is
our task as their designers and creators to build the information into
them in the form of works of art that makes it obvious how to repair and
extend the robot. A robot should also be constructed in such a way that
it is its own means of production: the components of the robot can be
used as a machine to build more robots like it. This will require human
effort, but both the physical tools and the information required to
learn the skills to duplicate the machine are built into the machine and
obvious to find and use. Modern technology is designed to scare you away
from modifying it or interacting with it in any deep way. We seek to
build machines that do the opposite: invite the user to get more deeply
involved, building more, documenting that process, and extending the
technology themselves for others to use.

I will illustrate this with an example. One of the simplest robots will
move itself around looking for energy, then when it finds some(generally
a fast moving water body like a creek) it will turn itself into both a
power plant and a chemical plant, storing energy and chemicals extracted
from the water(targeting human industrial waste of various kinds). This
will involve a computer, some motors and some sensors. Other machines
will be involved in large scale computer fabrication as outlined in
other sections of this work.

\subsection{Cyborgs}\label{cyborgs}

A cyborg, or cybernetic organism, is a combination of artificial devices
and living things. I believe that we should blur these lines both in
ourselves(we have already done that) and in our fellow living things
with whom we should be able to more harmoniously co exist.

One other thing to observe about both ourselves and our fellow living
things is that when examined closely we are almost all in a symbiotic
relationship with other living things. We need our gut bacteria to live,
cows need fungi and bacteria both, trees rely on fungi for their roots
to be robust, which ensures the survival of the whole forrest, etc. The
responsible development of cyborgs should combine living things with non
living things in a thoughtful and artistic and compassionate way.

I propose that the more communication oriented of our fellow living
things should be given access to our communication networks. I have no
idea what happens if you build a virtual reality headset for a octopus
or squid and allow them to communicate with other squids thousands of
miles away. But if they can put their heads in or not on their own,
surely it's worth trying? Perhaps we could allow them to smoothly join
our society, and co exist with us if we allowed them to communicate with
each other using our tools first. A truly symbiotic relationship with
cephalopods might end up with an arrangement where we help them live
good lives using our control of the oceans and our sensor systems for
weather and they use their bioluminescent skin as display technology for
some networked communication devices.

I would note that both for these water based creatures and for the
various flying creatures including insects, birds, and bats, their
brains may allow them to control flying drones with much better skill
than we can, and in large rumbles with much better flocking ability.
Perhaps another virtual reality rig is called for, allowing birds and
bats to connect with huge soaring drones so that they can expand their
minds the way we do with our machines.

We should not limit cyborg development to the obvious animals! Plants
and fungi and various strange micro life should be also investigated at
all levels. What would a slime mold cyborg look like? Something awesome,
one might hope. And finally plants, when integrated with technology, can
suddenly move! This leads us to the super ent, which is the next
section.

\subsection{Super Ents}\label{super-ents}

The fractal mater reactor should be alive. Trees, bushes, grass, etc.
can grow all around it, with roots going into various fractal channels
which provide nutrients. These liquid spaces can have various animals
and fungi and microorganisms, creating a whole ecosystem. Imagine an
island built up of such mater, the size of a small building, covered
with trees. Ambient energy is used to slowly build up and discharge
electrical energy to operate philosophy engines which slowly walk the
whole thing across the landscape. With little or even no human
intervention, this lumbering living giant might spend decades crawling
up and down hills scouring for junk cars, which it turns into an
ever-growing robot rumble that it can give away to any passing humans
for free at any time.

Building this kind of thing in the ocean can be incredibly powerful.
Whole floating islands filled with fractal reactor technology can wander
the high seas, with the humans all underwater in bubbles to ride out
storms, picking up storm energy and sea junk, and building a ever larger
floating city deep out in the ocean. This aquatic fractal techno city
could exist even in a dead world of violent storms and acid oceans and
extreme heat.

Machines that comb the ocean for contaminants, using waves go get energy
to move around and sort and grab stuff, potentially floating around for
years before being found based on a data stream that pulses out
periodically, and eventually another type of robot-tending robot can
grab it, extract the materials it's gathered, and bring it to a floating
factory robot rumble. This kind of robot is important for the ecosystem
of the jungle city in the ocean-inundated coastal post apocalypse.

\subsection{\ldots{}And a World To Win!}\label{and-a-world-to-win}

The Anthropocene is here. Like it or not, it's here. For the next 1000
years our planet is going to be dominated by the actions we choose to
take as a civilization. If we stay on the track we're on, the atmosphere
and oceans heat up, massive desertification destroys wet ecosystems
while rising oceans eat most of our cities, and the oceans become a
toxic waste dump that cannot sustain life. If we do nothing that is
clearly what will happen. Or something worse involving nuclear
holocaust. Given these alternatives, what difference does it make how
drastically we change things in the sea, air, and land? The opportunity
to simply not let civilization get big enough to destroy the world has
long passed us by now.

So is it so wrong to imagine the whole landscape filled with these
lumbering rumbles of rolling, slithering, hopping, and gliding robots?
Is it wrong to let them reproduce with human help, but with very little
labor-time, allowing groups of people to build endlessly expanding
rumble spheres around the world to create a world of total abundance? I
say that it is not wrong. Maybe if there were a way to go back it would
be a hard choice to do something that disturbs the balance of nature
like this, but there isn't.

This is what the trash wizard wants to make possible in the world:
endless streams of material and data moving through the physical world
with robots made from trash, which encompass our whole human
environment. Maybe not the whole world, but enough of it. A world of
abundance using the rumble sphere and value circles could exist outside
of the states and corporations. It does not need land, just someplace to
move to--it is all mobile by default. The trash wizards build the needed
expertise up and document it and teach it so that any group of people
can create this kind of culture anywhere, specific to their individual
cultural needs and the available resources in whatever geographical area
they're in.

\end{document}
